%%%%%%%%%%%%%%%%%%%%%%%%%%%%%%%%%%%%%%%%%
% Vertical Line Title Page 
% LaTeX Template
% Version 1.0 (27/12/12)
%
% This template has been downloaded from:
% http://www.LaTeXTemplates.com
%
% Original author:
% Peter Wilson (herries.press@earthlink.net)
%
% License:
% CC BY-NC-SA 3.0 (http://creativecommons.org/licenses/by-nc-sa/3.0/)
% 
% Instructions for using this template:
% This title page compiles as is. If you wish to include this title page in 
% another document, you will need to copy everything before 
% \begin{document} into the preamble of your document. The title page is
% then included using \titleGM within your document.
%
%%%%%%%%%%%%%%%%%%%%%%%%%%%%%%%%%%%%%%%%%

%----------------------------------------------------------------------------------------
%	PACKAGES AND OTHER DOCUMENT CONFIGURATIONS
%----------------------------------------------------------------------------------------

\documentclass{report}

\usepackage{cite,graphicx}
\usepackage[chapter]{algorithm}
\usepackage{url,float}
\usepackage{hyperref}
\usepackage{uithesis}
\usepackage{listings}
\usepackage[nodayofweek,level]{datetime}
\usepackage[bahasai]{babel}
\usepackage{subfigure}
\selectlanguage{bahasai}
\renewcommand{\bibname}{Daftar Referensi}
\renewcommand{\contentsname}{Daftar Isi}
\renewcommand{\listfigurename}{Daftar Gambar}
\renewcommand{\listtablename}{Daftar Tabel}
\renewcommand\lstlistingname{Program}
\renewcommand\lstlistlistingname{Daftar Program}
\renewcommand{\chaptername}{BAB}
\renewcommand{\figurename}{\bo{Gambar}}
\renewcommand{\tablename}{\bo{Tabel}}
\Var{\kataPengantar}{Kata Pengantar}
\include{hype.indonesia}

%----------------------------------------------------------------------------------------
%	TITLE PAGE
%----------------------------------------------------------------------------------------

\newcommand*{\titleGM}{\begingroup % Create the command for including the title page in the document
\hbox{ % Horizontal box
\hspace*{0.2\textwidth} % Whitespace to the left of the title page
\rule{1pt}{\textheight} % Vertical line
\hspace*{0.05\textwidth} % Whitespace between the vertical line and title page text
\parbox[b]{0.75\textwidth}{ % Paragraph box which restricts text to less than the width of the page

{\noindent\Huge\bfseries Pemrograman Python untuk Pengolahan Citra Digital }\\[2\baselineskip] % Title
{\large \textit{Diktat kuliah}}\\[4\baselineskip] % Tagline or further description
{\large \textsc{Dr. Arya Adhyaksa Waskita}} % Author name

\vspace{0.5\textheight} % Whitespace between the title block and the publisher
\begin{figure}[H]
\includegraphics[scale=.2]{pics/logo.png}
\end{figure}
STMIK Eresha - 2020
%Powered by \LaTex

%{\noindent The Publisher \plogo}\\[\baselineskip] % Publisher and logo
}}
\endgroup}

%----------------------------------------------------------------------------------------
%	BLANK DOCUMENT
%----------------------------------------------------------------------------------------

\begin{document}

%\addChapter{\kataPengantar}


% Removes page numbers

\titleGM % This command includes the title page
\pagenumbering{roman}
\setcounter{page}{0}
\tableofcontents
\listoffigures
\addChapter{Daftar Program}
\lstlistoflistings
\addChapter{\kataPengantar}
%-----------------------------------------------------------------------------%
\chapter*{Kata Pengantar}
%-----------------------------------------------------------------------------%
Diktat kuliah ini hanya merupakan pelengkap agar mahasiswa dapat lebih mudah memahami materi pengolahan citra digital. Penggunaan ilustrasi lain dari perangkat lunak berbayar dapat saja diberikan. Tetapi, karena pertimbangan kemandirian dan lisensi, maka saya memutuskan untuk menyusun diktat ini berbasis pada pustaka berlisensi publik dan berbasis bahasa pemrograman Python, \texttt{scikit-image}. Python dipertimbangkan karena banyak pustaka ilmiah yang sudah umum digunakan dan terus dikembangkan yang berbasis pada Python. Dalam pengolahan citra, selain \texttt{scikit-image}, ada juga \texttt{OpenCV} untuk \textit{Computer Vision}. Dalam pembelajaran mesin, \texttt{scikit-learn} adalah pustaka yang juga banyak digunakan. Bahkan \texttt{tensorflow}, pustaka yang banyak digunakan dalam penelitian \textit{deep learning} juga berbasis pada Python. Saya yakin, dengan mempelajari diktat ini, mahasiswa mampu mandiri dalam penguasaan bahasa pemrograman Python yang pada akhirnya mampu membuat mahasiwa lebih adaptif terhadap pustaka berbasis python, baik untuk tujuan ilmiah maupun bisnis. Mahasiswapun diharapkan menjadi lebih kreatif dalam melakukan penelitian hingga mengembangkan produk perangkat lunak, maupun prototipe perangkat keras cerdas berbasis Python tanpa harus terbebani masalah lisensi.

Secara umum, diktat ini dibagi ke dalam bagian pendahuluan yang membahas tentang sejarah singkat Python yang dilanjutkan ke bagian instalasi. Instalasi ini, meskipun sangat sederhana, terutama pada sistem operasi Linux, dapat menjadi sangat merepotkan bagi beberapa mahasiswa, terutama ketika mereka menggunakan sistem operasi Windows. Karena itu, instalasi akan dilakukan di sistem operasi Windows. Bagian selanjutnya adalah dasar-dasar pemrograman Python, terutama struktur data (\texttt{list}, \texttt{tuple} dan \texttt{dictionary}), interaksi dengan \textit{file}, hingga mempelajari penggunaan fungsi yang terdapat dalam pustaka tertentu. Sedangkan bagian terkahir dari diktat ini akan sepenuhnya diisi dengan fitur pustaka \texttt{scikit-image}, yang saat diktat ini disusun berada pada rilis \texttt{0.17.2}.
\clearpage
Diktat ini tidak ditujukan untuk menjadi rujukan dalam teknik pengolahan citra. Sehingga penjelasan teoritis terkait pengolahan citra akan diberikan dalam porsi yang sangat minim dan hanya ditujukan sebegai pelengkap saja. Selain itu, dalam diktat ini banyak menggunakan sumber dari situs web dan akan disampaikan secara detil alamat sumber tersebut dalam diktat. Diharapkan, mahasiswa tidak takut mencoba karena ada begitu banyak sumber yang dapat digunakan untuk belajar. Hanya kesungguhan kitalah yang akan menjadi pembeda. Akhirnya, selamat mencoba pengalaman baru. 

\vspace*{2cm}
\begin{flushright}
\selectlanguage{bahasai}
Serpong, \today\\[0.1cm]
\vspace*{1cm}
Dr. Arya Adhyaksa Waskita

\end{flushright}


\pagenumbering{arabic}
%\chapter{Sejarah Pemrograman Python}

\chapter{Instalasi Python}

\section{Interpreter Python}
\label{sec:interpreter}
Seperti telah dijelaskan di bagian Pengantar, instalasi \textit{interpreter} Python dilakukan di sistem operasi Windows 7. Tahapan instalasi ini mengasumsikan bahwa tidak ada kendala apapun terkait sistem operasi. Selanjutnya mahasiwa diminta untuk mengunduh \textit{interpreter} Python melalui laman \url{https://www.python.org/downloads/} sesuai kebutuhannya. 

Mengeksekusi unduhan tersebut akan memunculkan dialog seperti pada \figurename~\ref{fig:install1}. Pastikan untuk memilih konfigurasi \texttt{PATH} secara otomatis agar ketika proses instalasi selesai, \textit{interpreter} Python dapat dijalankan dari mana saja di sistem komputer masing-masing. Untuk kondisi di mana terjadi kesalahan, akan muncul dialog yang memberi kita kesempatan untuk melihat \textit{log}. Buka log tersebut dan lihat sumber dari kesalahan instalasi yang sedang terjadi.

\begin{figure}[h!]
   \begin{center}
     \includegraphics[scale=.5]{pics/installPython1.png}
     \caption{Dialog instalasi \textit{interpreter} Python}
     \label{fig:install1}
   \end{center}
 \end{figure} 

Pilihan opsi \textit{Customize installation} akan menampilkan dialog seperti \figurename~\ref{fig:feature}. Pastikan semua pilihan dipilih.

\begin{figure}[h!]
  \begin{center}
    \includegraphics[scale=.5]{pics/featureInstall.png}
    \caption{Pilihan paket pendukung sebelum instalasi dilakukan}
    \label{fig:feature}
  \end{center}
\end{figure}

Selama proses instalasi berlangsung, pengguna akan disuguhkan dialog seperti \figurename~\ref{fig:installProgres}. Tunggu sampai dialog tanda selesai dikeluarkan seperti pada \figurename~\ref{fig:finish}.

\begin{figure}[h!]
  \begin{center}
    \includegraphics[scale=.5]{pics/installProgress.png}
    \caption{Dialos selama proses instalasi berlangsung}
    \label{fig:installProgres}
  \end{center}
\end{figure}

\begin{figure}[h!]
  \begin{center}
    \includegraphics[scale=.5]{pics/installFinished.png}
    \caption{Dialog tanda selesai instalasi}
    \label{fig:finish}
  \end{center}
\end{figure}

Seperti telah ditunjukkan pada \figurename~\ref{fig:install1} tentang informasi lokasi \textit{interpreter} Python diletakkan, dapat juga dibuktikan melalui aplikasi \texttt{CMD} seperti \figurename~\ref{fig:lokasi}. Sedangkan \textit{interpreter} Python dapat diujicobakan dengan menuliskan perintah \texttt{python} di aplikasi \texttt{CMD}. Akan muncul dialog seperti \figurename~\ref{fig:siap}. \textit{Interpreter} Python siap digunakan, ditandai dengan munculnya karakter \texttt{>>>}.

\begin{figure}[h!]
  \begin{center}
    \includegraphics[scale=.5]{pics/installedLocation.png}
    \caption{Lokasi instalasi \textit{interpreter} Python}
    \label{fig:lokasi}
  \end{center}
\end{figure}

\begin{figure}[h!]
  \begin{center}
    \includegraphics[scale=.5]{pics/pythonAktif.png}
    \caption{\textit{Interpreter} Python siap digunakan}
    \label{fig:siap}
  \end{center}
\end{figure}

Tahapan selanjutnya adalah instalasi pustaka \texttt{scikit-image}. Proses instalasinya dilakukan dengan aplikasi pengelola paket Python yang bernama \texttt{pip}. Silakan lihat \figurename~\ref{fig:feature}. \texttt{pip} ada di urutan kedua dari fitur tambahan. \texttt{pip} dapat digunakan untuk melihat paket apa saja yang telah terpasang di sistem kita. Caranya dengan menjalankan perintah \texttt{python -m pip list} seperti ditunjukkan \figurename~\ref{fig:daftarPaket}.

\begin{figure}[h!]
  \begin{center}
    \includegraphics[scale=.5]{pics/pipList.png}
    \caption{Daftar paket yang terpasang}
    \label{fig:daftarPaket}
  \end{center}
\end{figure}

\texttt{pip} dapat juga digunakan untuk meng-\texttt{upgrade} paket yang telah terpasang, bahkan dirinya sendiri. Untuk meng-\textit{upgrade} paket \texttt{pip} itu sendiri, dapat dilakukan dengan menjalankan perintah \texttt{python -m pip install --upgrade pip} seperti \figurename~\ref{fig:pipUpgrade}. Perhatikan versi \texttt{pip} yang ada di \figurename~\ref{fig:daftarPaket} dan \figurename~\ref{fig:pipUpgrade}.
 
\begin{figure}
  \begin{center}
    \includegraphics[scale=.5]{pics/pipList2.png}
    \caption{Hasil \texttt{upgrade} pip}
    \label{fig:pipUpgrade}
  \end{center}
\end{figure}

Sedangkan untuk memasang pustaka \texttt{scikit-image}, jalankan perintah \texttt{python -m pip install scikit-image} pada aplikasi \texttt{CMD} seperti \figurename~\ref{fig:installSkimage}.

\begin{figure}[h!]
  \begin{center}
    \includegraphics[scale=.5]{pics/installScikit-Image.png}
    \caption{Instalasi pustaka \texttt{scikit-image} menggunakan \texttt{pip}}
    \label{fig:installSkimage}
  \end{center}
\end{figure}

Jika ada pustaka lain yang menjadi ketergantungan dari pustaka yang akan diinstal, pip akan melakukan instalasi secara otomatis. \figurename~\ref{fig:installDepend} menunjukkan proses tersebut. Hal ini akan sangat memudahkan pengguna mengelola pustaka Python yang digunakan.

\begin{figure}[h!]
  \begin{center}
    \includegraphics[scale=.5]{pics/installScikit-Imagedependencies.png}
    \caption{Instalasi pustaka \textit{dependent}}
    \label{fig:installDepend}
  \end{center}
\end{figure}

Setelah selesai, kita dapat kembali melihat daftar paket yang terpasang melalui pengelolaan \texttt{pip} yang ditunjukkan \figurename~\ref{fig:daftarPaket2}.

\begin{figure}[h!]
  \begin{center}
    \includegraphics[scale=.5]{pics/pipList3.png}
    \caption{Daftar terakhir paket terpasang}
    \label{fig:daftarPaket2}
  \end{center}
\end{figure}

Menu aplikasi pendukung Python akan muncul seperti \figurename~\ref{fig:menu}. Menu kedua pada \figurename~\ref{fig:menu} akan memunculkan aplikasi \texttt{CMD} yang sama dengan yang ditunjukkan \figurename~\ref{fig:siap}, tetapi tanpa perlu memanggil perintah \texttt{python} terlebih dahulu. CMD secara otomatis akan memunculkan Python \texttt{shell} seperti \figurename~\ref{fig:siap}.

\begin{figure}[h!]
  \begin{center}
    \includegraphics[scale=.5]{pics/menuPython.png}
    \caption{Daftar menu aplikasi pendukung Python}
    \label{fig:menu}
  \end{center}
\end{figure}

\texttt{IDLE} adalah antarmukan \textit{interpreter} Python seperti ditunjukkan \figurename~\ref{fig:idle}. Dalam \figurename~\ref{fig:idle} juga terlihat bahwa kita berhasil meng-\textit{import} pustaka \texttt{scikit-image}, yang dalam \texttt{IDLE} di Windows 7 disebut sebagai \texttt{skimage}. Jika Anda sedang menggunakan Ubuntu, kemudian menggunakan pustaka \texttt{scikit-image} yang diperoleh dari \textit{repository} Ubuntu (bukan dari \texttt{pip}), pustaka \texttt{scikit-image} juga di-\textit{import} dengan nama \texttt{skimage}. Berhasilnya sebuah pustaka Python di-\textit{import} adalah ketika tidak ada komentar yang muncul setelah perintah \texttt{import} tersebut.

\begin{figure}
  \begin{center}
    \includegraphics[scale=.5]{pics/idle.png}
    \caption{Aplikasi \texttt{IDLE}}
    \label{fig:idle}
  \end{center}
\end{figure}

Selanjutnya, jika ditemukan petunjuk untuk masuk ke Python \texttt{Shell}, Anda dapat menggunakan aplikasi \texttt{IDLE}\texttt{}, atau menggunakan terminal (di Linux)/\texttt{CMD} (di Windows) dengan terlebih dahulu menjalankan perintah \texttt{python}.

\section{Anaconda}
Selain pilihan manual seperti yang telah dijelaskan di Sub bab \ref{sec:interpreter}, Anaconda bisa menjadi opsi lain yang lebih bersifat otomatis. Saya menyebutnya otomatis karena Anaconda sejumlah pustaka Python, terutama yang banyak digunakan di \textit{Data Mining}, \textit{Machine Learning} atau \textit{Data Science} telah dikemas di dalam Anaconda. Bahkan beberapa editor yang populer untuk Python juga dikemasnya. Anaconda bahkan mengemasnya khusus untuk \textit{platform} yang berbeda. Anda dapat menghubungi alamat \url{https://www.anaconda.com/} untuk mengunduh aplikasinya. Sesuaikan kebutuhan Anda dengan pilihan yang ada seperti ditunjukkan \figurename~\ref{fig:platformAnaconda}.

\begin{figure}[h!]
  \begin{center}
    \includegraphics[scale=.5]{pics/anacondaInstall0.png}
    \caption{Pilihan \textit{platform} instalasi Anaconda}
    \label{fig:platformAnaconda}
  \end{center}
\end{figure}

Instalasi Anaconda akan menghadirkan dialog seperti ditunjukkan \figurename~\ref{fig:pembuka} - \figurename~\ref{fig:instalasiEnd}.

\begin{figure}[h!]
  \begin{center}
    \includegraphics[scale=.5]{pics/anacondaInstall1.png}
    \caption{Dialog pembuka instalasi}
    \label{fig:pembuka}
  \end{center}
\end{figure}

\begin{figure}[h!]
  \begin{center}
    \includegraphics[scale=.5]{pics/anacondaInstall2.png}
    \caption{Menyetujui kesepakatan}
    \label{fig:kesepakatan}
  \end{center}
\end{figure}

\begin{figure}[h!]
  \begin{center}
    \includegraphics[scale=.5]{pics/anacondaInstall3.png}
    \caption{Pilihan pengguna Anaconda}
    \label{fig:pengguna}
  \end{center}
\end{figure}

\begin{figure}[h!]
  \begin{center}
    \includegraphics[scale=.5]{pics/anacondaInstall4.png}
    \caption{Target instalasi}
    \label{fig:target}
  \end{center}
\end{figure}

\begin{figure}[h!]
  \begin{center}
    \includegraphics[scale=.5]{pics/anacondaInstall5.png}
    \caption{Menjadikan Anaconda sebagai sistem utama Python}
    \label{fig:utama}
  \end{center}
\end{figure}

\begin{figure}[h!]
  \begin{center}
    \includegraphics[scale=.5]{pics/anacondaInstall6.png}
    \caption{Proses instalasi}
    \label{fig:prosesInstalasi}
  \end{center}
\end{figure}

\begin{figure}[h!]
  \begin{center}
    \includegraphics[scale=.5]{pics/anacondaInstall9.png}
    \caption{Instalasi selesai}
    \label{fig:instalasiEnd}
  \end{center}
\end{figure}

Instalasi Anaconda akan membuat menu seperti pada \figurename~\ref{fig:menuAnaconda}. Di situ terlihat sejumlah aplikasi yang dapat digunakan untuk mengembangkan kode komputer berbasis Python seperti Jupyter dan Spyder. Untuk Jupyter, aplikasi ini akan menghadirkan antarmuka seperti tampak pada \figurename~\ref{fig:jupyter}. Di sisi kanan atas terlihat beberapa opsi antarmuka untuk mengelola proyek Python dengan Jupyter, seperti Terminal \figurename~\ref{fig:jupyterTerminal} atau Python \texttt{Shell} di bawah Jupyter seperti \figurename~\ref{fig:jupyterShell} yang perannya seperti \texttt{IDLE} di \figurename~\ref{fig:idle}. Sedangkan untuk Spyder, akan tampak antarmuka seperti \figurename~\ref{fig:spyder}.

\begin{figure}[h!]
  \begin{center}
    \includegraphics[scale=.5]{pics/anacondaMenu2.png}
    \caption{}
    \label{fig:menuAnaconda}
  \end{center}
\end{figure}

\begin{figure}[h!]
  \begin{center}
    \includegraphics[scale=.5]{pics/jupyter2.png}
    \caption{Aplikasi \texttt{Jupyter}}
    \label{fig:jupyter}
  \end{center}
\end{figure}

\begin{figure}[h!]
  \begin{center}
    \includegraphics[scale=.5]{pics/jupyter3.png}
    \caption{Terminal pada aplikasi \texttt{Jupyter}}
    \label{fig:jupyterTerminal}
  \end{center}
\end{figure}

\begin{figure}[h!]
  \begin{center}
    \includegraphics[scale=.5]{pics/jupyter4.png}
    \caption{Python \texttt{Shell} pada aplikasi \texttt{Jupyter}}
    \label{fig:jupyterShell}
  \end{center}
\end{figure}

\begin{figure}[h!]
  \begin{center}
    \includegraphics[scale=.45]{pics/spyder.png}
    \caption{Aplikasi \texttt{Spyder}}
    \label{fig:spyder}
  \end{center}
\end{figure}


\chapter{Dasar Pemrograman Python}
\section{Pendahuluan}
Bahasa pemrograman Python memiliki 4 sifat dasar berikut\footnote{\url{https://www.tutorialspoint.com/python/index.htm}}.
\begin{enumerate}
  \item \textit{Interpreter}. Python diproses oleh \textit{interpreter}, sehingga tidak perlu dikompilasi untuk menjalankannya. Hal ini seperti dijumpai pada bahasa pemrograman PHP yang sangat populer itu.
  \item Interaktif. Anda dapat berinteraksi denga Python dengan memberikannya perintah satu per satu melalui Python \texttt{shell}. Setiap perintah yang diberikan langsung akan direspon. Selain itu, Python bersifat \textit{self explained}. Jika ada fungsi dari suatu obyek yang tidak kita ketahui, kita bisa mempelajarinya langsung dari dokumentasi di Python \texttt{shell}.
  \item Berorientasi obyek. Ada semacam slogan bahwa '''\textit{Everything is object in Python}'''. Seperti telah dipahami melalu kuliah Rekayasa Perangkat Lunak, orientasi obyek menyebabkan variabel dan fungsi (sering disebut sebagai \textit{state} dan \textit{behavior}) terkemas dalam sebuah obyek, sehingga memudahkan pengelolaan variabel. Fungsi yang melekat pada sebuah obyek juga dapat diturunkan dari satu obyek ke obyek lain sehingga tidak perlu dideklarasi ulang. Namun, fitur orientasi obyek ini pemberlakuannya bagi pemrogram tidak seketat seperti yang dilakukan di \texttt{Java}. Jika \texttt{Java} mengharuskan pemrogram mendeklarasikan kelas untuk membuat program yang bahkan sangat sederhana, makan Python tidak mengharuskannya.
  \item Bahasa pemrograman untuk pemula. Hal ini disebabkan karena Python sangat sederhana, tidak memerlukan banyak deklarasi yang seringkali menyulitkan, bahkan menakutkan bagi pemula. Selain itu, Python juga mendukung pengembangan aplikasi untuk banyak \textit{platform}, dari aplikasi \textit{embedded} hingga \textit{web} dan \textit{mobile}. 
\end{enumerate}

Untuk sifat dasar pertama dan kedua, dapat dilihat ilustrasinya di \figurename~\ref{fig:interpreter}. Dalam \figurename~\ref{fig:interpreter}, Python \texttt{shell} dipanggil dengan perintah \texttt{python3}. Hal tersebut disebabkan karena Ubuntu (yang sedang digunakan adalah Ubuntu 18.04) secara \textit{default} menyertakan Python versi 2.x. Sedangkan untuk Python versi 3.x harus dijalankan dengan perintah \texttt{python3}. Di \figurename~\ref{fig:interpreter} terlihat bahwa ada dua perintah yang diberikan secara berurutan. Tetapi, Python akan meresponnya satu per satu. Sedangkan untuk keluar dari Python \texttt{shell}, berikan perintah \texttt{exit()}.

\begin{figure}[h!]
  \begin{center}
    \includegraphics[scale=.5]{pics/interpreter.png}
    \caption{Python \texttt{shell} sedang menerima perintah}
    \label{fig:interpreter}
  \end{center}
\end{figure}

Untuk sifat dasar ketiga dapat diilustrasikan melalui \figurename~\ref{fig:obyek}. Kita dapat mengetahui jenis obyek dari variabel \texttt{a} dengan fungsi \texttt{type(a)}. Sedangkan untuk melihat fungsi dan variabel apa saja yang terkandung pada variabel \texttt{a}, kita dapat menggunakan fungsi \texttt{dir(a)}. Tetapi, meskipun semuanya di dalam Python adalah obyek, penggunaan Python tidak mengharuskan kita mendeklarasi kelas secara eksplisit. Dengan menuliskan perintah \texttt{a=3}, Python tahu bahwa obyek \texttt{a} adalah obyek dari kelas \texttt{integer}. Bahkan, di \figurename~\ref{fig:interpreter}, operasi aritmatika dapat dilakukan tanpa mendeklrasi variabel.

\begin{figure}[h!]
  \begin{center}
    \includegraphics[scale=.5]{pics/interpreter2a.png}
    \caption{Variabel \texttt{a} sebagai obyek}
    \label{fig:obyek}
  \end{center}
\end{figure}

Di \figurename~\ref{fig:obyek} terlihat ada entitas yang diawali dan/atau diakhir dengan karakter dua \textit{underscore} ('\_\_') atau sering disebut sebagi \textit{dunder}\footnote{\url{https://dbader.org/blog/meaning-of-underscores-in-python}} (\textit{double undescore}) oleh komunitas pemrogram Python. Hal tersebut merupakan bagian dari PEP (\textit{Python Enhancement Proposals}) ke-8 tentang \textit{Style Guide for Python Code}\footnote{\url{https://www.python.org/dev/peps/pep-0008/}}.

Di \figurename~\ref{fig:obyek} juga terlihat bahwa obyek \texttt{a} memiliki fungsi \texttt{\_\_doc\_\_}. Fungsi inilah yang akan memberikan penjelasan singkat kepada kita tentang obyek yang sedang menjadi perhatian. Untuk menggunakannya, jalankan perintah \texttt{a.\_\_doc\_\_} seperti ditunjukkan \figurename~\ref{fig:doc}. Dengan \texttt{a} adalah nama variabel untuk obyek yang sedang menjadi perhatian.

\begin{figure}[h!]
  \begin{center}
    \includegraphics[scale=.5]{pics/interpreter3.png}
    \caption{Menampilkan dokumentasi obyek \texttt{integer a}}
    \label{fig:doc}
  \end{center}
\end{figure}

Format dokumentasi seperti yang ditunjukkan pada \figurename~\ref{fig:doc} sulit untuk dipahami. Pendekatan lain untuk mempelajari dokumentasi sebuah pustaka adalah dengan menggunakan fungsi \texttt{help}. Untuk kasus seperti \figurename~\ref{fig:doc}, perintah yang dijalankan adalah \texttt{help(a)} (\textbf{BUKAN} \texttt{a.\_\_doc\_\_}). Hasilnya ditunjukkan pada \figurename~\ref{fig:doc2}. Untuk keluar dari modus dokumentasi tersebut, pengguna tinggal memberi perintah \texttt{q} setelah tanda titik dua (\figurename~\ref{fig:doc2}). Sedangkan untuk melihat isi dokumentasi selanjutnya pengguna dapat menggunkana tombol spasi di papan ketik.

\begin{figure}
  \begin{center}
    \includegraphics[scale=.5]{pics/interpreter4.png}
    \caption{Menampilkan dokumentasi obyek \texttt{integer a} menggunakan fungsi \texttt{help}}
    \label{fig:doc2}
  \end{center}
\end{figure}


\chapter{Pustaka \texttt{Scikit-Image}}
\section{Pendahuluan}

Saat diktat ini disusun, versi stabil terbaru dari pustaka \texttt{scikit-image} adalah \texttt{0.16.2}. Diktat ini disusunan berdasarkan penjelasan yang disajikan di \url{https://scikit-image.org/}. Sedangkan alur penyajiannya didasarkan pada kebutuhan untuk mendapatkan fitur citra.

Seperti dijelaskan \cite{Gonzalez} pada \figurename~\ref{fig:targetProcessing}, pengolahan citra mentargetkan kemampuan pengenalan obyek. \textit{Image enhancement} dan \textit{Image restoration} digunakan untuk mendapatkan fitur citra yang optimal. Hal ini disebabkan karena pada kondisi tertentu, citra mengandung banyak sekali \textit{noise} yang menyebabkan fiturnya sulit diekstraksi. Hal ini dapat membuat pengenalan obyek di dalam citra tidak maksimal. 

\textit{Enhancement} dan \textit{Restoration} pada citra dapat dilakukan pada domain spasial maupun frekuensi. Pada domain spasial, citra diperlakukan seperti apa adanya, yaitu matriks dengan ukuran sebanyak piksel penyusun, yang berisi intensitas warna pada setiap element matriks. Sedangkan untuk domain frekuensi, citra dianggap sebagai representasi sejumlah gelombang elektromagnetik dengan beragam frekuensi yang menjadi satu. Komponen berfrekuensi tinggi direpresentasi oleh gradasi intensitas warna yang cepat pada domain spasial. Sebaliknya, komponen berfrekuensi rendah direpresentasikan oleh gradisi intensitas warna yang lambat pada domain spasial. \textit{Enhancement} dan \textit{Restoration} citra dapat dilakukan menggunakan transformasi Fourier maupun wavelet (\figurename~\ref{fig:targetProcessing}).

Tahapan ekstraksi fitur yang tidak menjadi fokus pada diktat ini berdasarkan \figurename~\ref{fig:targetProcessing} adalah kompresi. Yang mungkin masih dapat dikategorikan sebagai kompresi feature selection yang merupakan pemilihan fitur hasil ekstraksi yang paling dominan dalam mencirikan suatu obyek di dalam citra. Tetapi, jika yang dimaksud adalah kompresi citra dari sudut pandang ukuran, maka hal tersebut tidak dibahas dalam diktat ini. Kompresi citra untuk mengurangi ukuran, baik untuk mengefisienkan media penyimpanan maupun jalur komunikasi sudah tidak menjadi fokus para peneliti saat ini. Selain karena kapasitas media penyimpanan dan \textit{bandwidth} komunikasi yang semakin besar dan semakin murah, kompresi ukuran citra yang tidak tepat dapat mengurangi informasi penting yang dapat menjadi fitur citra tersebut. Akibatnya, kemampuan pengenalan obyek dalam citra menurun.

Terakhir, fitur yang berhasil diekstraksi dari berbagai metode pengolahan citra akan menjadi masukan bagi pustaka Python lain seperti \texttt{scikit-learn} dan \texttt{tensorflow}.

\begin{figure}[h!]
  \begin{center}
    \includegraphics[scale=.5]{pics/steps.png}
    \caption{Pengeolahan citra untuk pengenalan obyek \cite{Gonzalez}}
    \label{fig:targetProcessing}
  \end{center}
\end{figure}

\section{Sub modul I/O}
Penjelasan tentang pengolahan citra berbasis \texttt{scikit-image} akan dimulai dengan sub module I/O (\textit{Input}/\textit{Output}). Pengguna harus memahami cara \texttt{scikit-image} membaca sebuah citra dan representasi dari pembacaan tersebut dalam komputer. Sebagai ilustrasi, citra uji berupa hewan \textit{baboon} \footnote{\url{https://homepages.cae.wisc.edu/~ece533/images/baboon.png}} ditunjukkan pada \figurename~\ref{fig:baboon}. 

\begin{figure}[h!]
  \begin{center}
    \includegraphics{pics/baboon.png}
    \caption{Citra uji \textit{baboon}}
    \label{fig:baboon}
  \end{center}
\end{figure}

\figurename~\ref{fig:baboon} berukuran \texttt{512x512} piksel yang berarti akan ada 3 matriks berukuran \texttt{512x512}, masing-masing untuk warna merah, hijau dan biru. Setiap elemen matriks akan bernilai integer di antara \texttt{0} dan \texttt{255}. Untuk membaca citra digital, digunakan fungsi \texttt{imread}, sebuah fungsi yang terdefinisi di bawah sub modul \texttt{scikit-image/io}. Masukkan perintah \lstlistingname~\ref{lst:imread} berikut di Python \texttt{shell} seperti \figurename~\ref{fig:siap}. 

\scriptsize
\begin{lstlisting}[language=python, numbers=left, numberstyle=\tiny, caption=Membaca/membuka citra, showstringspaces=false, label=lst:imread]
>>> from skimage import io
>>> img=io.imread('baboon.png')
>>> type(img)
<class 'numpy.ndarray'>
>>> img.shape
(512, 512, 3)
>>> img2=io.imread('baboon.png', True)
>>> img2.shape
(512, 512)
>>> io.imsave('baboonGS.png', img2)

\end{lstlisting}
\normalsize

Perintah di baris ke-1 menunjukkan cara untuk meng-\textit{import} pustaka \texttt{io}. Di sistem operasi Windows\textregistered, lokasi pustakanya ditunjukkan di \figurename~\ref{fig:install1}. Sedangkan di sistem operasi GNU-Linux, lokasi pustakanya berada di \texttt{/home/arya/.local/lib/python3.6/site-packages/skimage}. Di bawahnya, terdapat struktur directory seperti ditunjukkan \figurename~\ref{fig:libLocation}. Terlihat bahwa \texttt{io} adalah \textit{sub directory} yang membuat cara pemanggilan pustaka adalah seperti baris ke-1 pada \lstlistingname~\ref{lst:imread}. Cara lainnya adalah dengan mengganti perintah di baris ke-1 dengan \texttt{import skimage.io}. \textit{Directory} seperti yang ditunjukkan \figurename~\ref{fig:libLocation} sama dengan daftar sub modul dari pustaka \texttt{scikit-image}\footnote{\url{https://scikit-image.org/docs/stable/api/api.html}}. Karenanya, pola pemanggilan pustaka juga memiliki pola yang sama dengan \texttt{io}.

\begin{figure}[h!]
  \begin{center}
    \includegraphics[scale=.65]{pics/libLocation.png}
    \caption{Berkas yang berada di dalam \textit{directory} \texttt{skimage}}
    \label{fig:libLocation}
  \end{center}
\end{figure}

Untuk baris ke-2 \lstlistingname~\ref{lst:imread}, ditunjukkan cara untuk menggunakan fungsi \texttt{imread}. Karena pustaka \texttt{io} di-\textit{import} menggunakan perintah \texttt{from skimage import io}, maka fungsi \texttt{imread} digunakan seperti pada baris ke-2. Jika pustaka \texttt{io} di-\textit{import} dengan perintah \texttt{import skimage.io}, maka fungsi \texttt{imread} digunakan dengan perintah \texttt{img=skimage.io.imread('baboon.png')}. Perlu diperhatikan, cara pembacaan citra seperti baris ke-2 hanya untuk kondisi di mana citra \texttt{baboon.png} berada pada \textit{directory} yang sama dengan lokasi Python \texttt{shell} dipanggil. Variabel \texttt{img} pada baris ke-2 menunjukkan pointer ke citra yang dibaca.

Jenis data dari variabel \texttt{img} diketahui dengan cara seperti ditunjukkan pada baris ke-3. Terlihat bahwa \texttt{img} merupakan variabel \texttt{numpy array}. Sedangkan untuk mengetahui ukuran dari \texttt{numpy array} digunakan perintah pada baris ke-5. Terlihat bahwa variabel \texttt{img} adalah 3 buah matriks berdimensi dua berukuran \texttt{512x512}. Hal ini menunjukkan bahwa citra yang sedang dibaca terdiri dari 3 komponen warna, masing-masing adalah R (\textit{Red}), G (\textit{Green}), dan B (\textit{Blue}). 

Untuk mengakses komponen warna tertentu (merah, hijau atau biru), gunakan perintah \texttt{img[:,:,0]} untuk komponen warna merah serta \texttt{img[:,:,1]} dan \texttt{img[:,:,2]} masing untuk komponen warna hijau dan biru. Pola akses matriksnya sama dengan apa yang dilakukan pada Matlab\textregistered.

Untuk membaca citra dalam bentuk skala keabuan, berikan perintah seperti baris ke-7. Baris ke-9 menunjukkan bahwa citra yang dibaca telah dikonversi ke dalam skala keabuan sehingga hanya terdiri dari 1 matriks berukuran \texttt{512x512}.

Untuk menyimpan citra yang tadi dibaca dalam bentuk skala keabuan, dapat digunakan perintah di baris ke-10. Argumen pertama (\texttt{'baboonGS.png'}) adalah nama berkas citra yang akan disimpan, sedangkan argumen kedua (\texttt{img2}) adalah matriks citra dalam skala keabuan. Hasilnya ditunjukkan pada \figurename~\ref{fig:baboonGS}.

\begin{figure}[h!]
  \begin{center}
    \includegraphics[scale=.125]{pics/baboonGS.png}
    \caption{Citra skala keabuan}
    \label{fig:baboonGS}
  \end{center}
\end{figure}

Sampai di sini, pustaka \texttt{numpy} tidak dibahas secara detil. Bagi yang tertarik dapat mempelajarinya secara daring di alamat \url{https://numpy.org/}. Untuk melihat fungsi apa saja yang dapat dilakukan oleh obyek \texttt{numpy} dapat diketahui dengan memberikan perintah \texttt{dir(img)} di Python \texttt{shell}, dengan \texttt{img} adalah obyek dari kelas \texttt{numpy}. 


\include{bab4}
\include{bab5}
%\chapter{Sub Modul Pustaka \texttt{Scikit-Image}}

\bibliographystyle{apalike}
\bibliography{reference}
\end{document}
