\chapter{Dasar Pemrograman Python}
\section{Pendahuluan}
Bahasa pemrograman Python memiliki 4 sifat dasar berikut\footnote{\url{https://www.tutorialspoint.com/python/index.htm}}.
\begin{enumerate}
  \item \textit{Interpreter}. Python diproses oleh \textit{interpreter}, sehingga tidak perlu dikompilasi untuk menjalankannya. Hal ini seperti dijumpai pada bahasa pemrograman PHP yang sangat populer itu.
  \item Interaktif. Anda dapat berinteraksi denga Python dengan memberikannya perintah satu per satu melalui Python \texttt{shell}. Setiap perintah yang diberikan langsung akan direspon. Selain itu, Python bersifat \textit{self explained}. Jika ada fungsi dari suatu obyek yang tidak kita ketahui, kita bisa mempelajarinya langsung dari dokumentasi di Python \texttt{shell}.
  \item Berorientasi obyek. Ada semacam slogan bahwa '''\textit{Everything is object in Python}'''. Seperti telah dipahami melalu kuliah Rekayasa Perangkat Lunak, orientasi obyek menyebabkan variabel dan fungsi (sering disebut sebagai \textit{state} dan \textit{behavior}) terkemas dalam sebuah obyek, sehingga memudahkan pengelolaan variabel. Fungsi yang melekat pada sebuah obyek juga dapat diturunkan dari satu obyek ke obyek lain sehingga tidak perlu dideklarasi ulang. Namun, fitur orientasi obyek ini pemberlakuannya bagi pemrogram tidak seketat seperti yang dilakukan di \texttt{Java}. Jika \texttt{Java} mengharuskan pemrogram mendeklarasikan kelas untuk membuat program yang bahkan sangat sederhana, makan Python tidak mengharuskannya.
  \item Bahasa pemrograman untuk pemula. Hal ini disebabkan karena Python sangat sederhana, tidak memerlukan banyak deklarasi yang seringkali menyulitkan, bahkan menakutkan bagi pemula. Selain itu, Python juga mendukung pengembangan aplikasi untuk banyak \textit{platform}, dari aplikasi \textit{embedded} hingga \textit{web} dan \textit{mobile}. 
\end{enumerate}

Untuk sifat dasar pertama dan kedua, dapat dilihat ilustrasinya di \figurename~\ref{fig:interpreter}. Dalam \figurename~\ref{fig:interpreter}, Python \texttt{shell} dipanggil dengan perintah \texttt{python3}. Hal tersebut disebabkan karena Ubuntu (yang sedang digunakan adalah Ubuntu 18.04) secara \textit{default} menyertakan Python versi 2.x. Sedangkan untuk Python versi 3.x harus dijalankan dengan perintah \texttt{python3}. Di \figurename~\ref{fig:interpreter} terlihat bahwa ada dua perintah yang diberikan secara berurutan. Tetapi, Python akan meresponnya satu per satu. Sedangkan untuk keluar dari Python \texttt{shell}, berikan perintah \texttt{exit()}.

\begin{figure}[h!]
  \begin{center}
    \includegraphics[scale=.5]{pics/interpreter.png}
    \caption{Python \texttt{shell} sedang menerima perintah}
    \label{fig:interpreter}
  \end{center}
\end{figure}

Untuk sifat dasar ketiga dapat diilustrasikan melalui \figurename~\ref{fig:obyek}. Kita dapat mengetahui jenis obyek dari variabel \texttt{a} dengan fungsi \texttt{type(a)}. Sedangkan untuk melihat fungsi dan variabel apa saja yang terkandung pada variabel \texttt{a}, kita dapat menggunakan fungsi \texttt{dir(a)}. Tetapi, meskipun semuanya di dalam Python adalah obyek, penggunaan Python tidak mengharuskan kita mendeklarasi kelas secara eksplisit. Dengan menuliskan perintah \texttt{a=3}, Python tahu bahwa obyek \texttt{a} adalah obyek dari kelas \texttt{integer}. Bahkan, di \figurename~\ref{fig:interpreter}, operasi aritmatika dapat dilakukan tanpa mendeklrasi variabel.

\begin{figure}[h!]
  \begin{center}
    \includegraphics[scale=.5]{pics/interpreter2a.png}
    \caption{Variabel \texttt{a} sebagai obyek}
    \label{fig:obyek}
  \end{center}
\end{figure}

Di \figurename~\ref{fig:obyek} terlihat ada entitas yang diawali dan/atau diakhir dengan karakter dua \textit{underscore} ('\_\_') atau sering disebut sebagi \textit{dunder}\footnote{\url{https://dbader.org/blog/meaning-of-underscores-in-python}} (\textit{double undescore}) oleh komunitas pemrogram Python. Hal tersebut merupakan bagian dari PEP (\textit{Python Enhancement Proposals}) ke-8 tentang \textit{Style Guide for Python Code}\footnote{\url{https://www.python.org/dev/peps/pep-0008/}}.

Di \figurename~\ref{fig:obyek} juga terlihat bahwa obyek \texttt{a} memiliki fungsi \texttt{\_\_doc\_\_}. Fungsi inilah yang akan memberikan penjelasan singkat kepada kita tentang obyek yang sedang menjadi perhatian. Untuk menggunakannya, jalankan perintah \texttt{a.\_\_doc\_\_} seperti ditunjukkan \figurename~\ref{fig:doc}. Dengan \texttt{a} adalah nama variabel untuk obyek yang sedang menjadi perhatian.

\begin{figure}[h!]
  \begin{center}
    \includegraphics[scale=.5]{pics/interpreter3.png}
    \caption{Menampilkan dokumentasi obyek \texttt{integer a}}
    \label{fig:doc}
  \end{center}
\end{figure}

Format dokumentasi seperti yang ditunjukkan pada \figurename~\ref{fig:doc} sulit untuk dipahami. Pendekatan lain untuk mempelajari dokumentasi sebuah pustaka adalah dengan menggunakan fungsi \texttt{help}. Untuk kasus seperti \figurename~\ref{fig:doc}, perintah yang dijalankan adalah \texttt{help(a)} (\textbf{BUKAN} \texttt{a.\_\_doc\_\_}). Hasilnya ditunjukkan pada \figurename~\ref{fig:doc2}. Untuk keluar dari modus dokumentasi tersebut, pengguna tinggal memberi perintah \texttt{q} setelah tanda titik dua (\figurename~\ref{fig:doc2}). Sedangkan untuk melihat isi dokumentasi selanjutnya pengguna dapat menggunkana tombol spasi di papan ketik.

\begin{figure}
  \begin{center}
    \includegraphics[scale=.5]{pics/interpreter4.png}
    \caption{Menampilkan dokumentasi obyek \texttt{integer a} menggunakan fungsi \texttt{help}}
    \label{fig:doc2}
  \end{center}
\end{figure}

\section{Operasi Berkas}

Yang dimaksud sebagai operasi berkas di sub bab ini ditujukan untuk memberikan solusi otomatis melakukan operasi pengolahan citra pada sejumlah besar citra (khususny), atau berkas digital secara umum. Sebagai ilustrasi, dataset citra terkait tumbuhan salah satunya dapat diperoleh di PlantCLEF2017 \footnote{\url{http://otmedia.lirmm.fr/LifeCLEF/PlantCLEF2017/TrainPackages/PlantCLEF2017Train1EOL.tar.gz}}.

Ketika berkas tersebut diekstrak, kita akan memperoleh \texttt{directory} data sebagai \texttt{directory} teratas dari dataset. Di dalamnya ada cukup banyak \texttt{directory} yang diberi nama berupa deretan angka. Di dalam \texttt{directory} tersebut, akan ada pasangan berkas dengan nama sama dari jenis \texttt{xml} dan \texttt{jpg}. Isi dari berkas berekstensi \texttt{xml} ditunjukkan oleh \lstlistingname~\ref{lst:135788}. Berkas ini berada dalam \texttt{directory} \texttt{9982}.

\scriptsize
\lstinputlisting[language=xml, numbers=left, numberstyle=\tiny, caption=135788.xml, showstringspaces=false, label=lst:135788]{script/135788.xml}
\normalsize

Yang perlu diperhatikan dari \texttt{element} berkas \texttt{xml} di \lstlistingname~\ref{lst:135788} adalah sebagai berikut.
\begin{itemize}
  \item \texttt{FileName}: elemen ini berisi informasi nama berkas
  \item \texttt{Content}: elemen ini berisi informasi jenis citra. \lstlistingname~\ref{lst:135788} menunjukkan bahwa citra yang sedang diamati adalah bunga.
  \item \texttt{Family}, \texttt{Genus}, \texttt{Species}: masing-masing menunjukkan tingkatan taksonomi dari tumbuhan. Dalam konteks klasifikasi tanaman, umumnya informasi spesies yang diperlukan. Tetapi, karena berkas pada dataset tidak tersusun dalam taksonomi, maka informasi yang disimpan pada elemen tersebut bermanfaat ketika kita ingin menyusun ulang struktur berkas citra berdasarkan taksonominya.
\end{itemize}

Operasi berkas yang dicontohkan dalam diktat ini adalah menyusun ulang citra berdasarkan jenis citra, apakah itu bunga atau daun. Kemudian di dalam \textit{directory} jenis citra tersebut, citra akan disusun mengikuti hirarki taksonominya. Sehingga akan ada struktur \textit{directory} seperti \texttt{Flower/Compositae/Carthamus/Carthamus\ caeruleus\ L.}. Pada kasus ini, berkas-berkas di dalamnya merupakan bunga dari \texttt{Family Compositae}, \texttt{Genus Carthamus} dan \texttt{Species Carthamus\ caeruleus\ L.}. Programnya disajikan pada \lstlistingname~\ref{lst:pynonxmlreader}. Di dalamnya ada sejumlah operasi berkas seperti pencarian dalam \textit{directory} tertentu, pencarian berkas dengan ekstensi tertentu sampai membuat \textit{directory} dan menduplikasi berkas dari satu \textit{directory} ke \textit{directory} lain.

\scriptsize
\lstinputlisting[language=python, numbers=left, numberstyle=\tiny, caption=Menysun ulang struktur berkas, showstringspaces=false, label=lst:pynonxmlreader]{script/pynonxmlreader.py}
\normalsize

Di \lstlistingname~\ref{lst:pynonxmlreader} disediakan juga pendeteksi kesalahan \texttt{try-except} ketika operasi berkas dilakukan. Hal ini dimaksudkan agar jalannya program tidak terhenti ketika kesalahan terjadi. Pengguna cukup mengetahui dari pesan kesalahan yang didefinisikan.

Jika fokus penelitian kita hanya pada bunga, maka kita hanya akan melakukan pengolahan pada citra yang berisi obyek bunga. Demikian juga untuk \texttt{Family}, \texttt{Genus} maupun \texttt{Species} tertentu.

Pada kondisi tertentu, misalnya seperti ditunjukkan \lstlistingname~\ref{lst:301518}, tidak ada informasi jenis citra. Hal ini terlihat dari tidak adanya isi elemen \texttt{Content}. Untuk citra yang salah satu atau seluruh elemennya tidak bernilai, \lstlistingname~\ref{lst:pynonxmlreader} akan mengelompokkannya sebagai \texttt{Undefined}. Untuk citra yang dideskripsikan oleh berkas \texttt{xml} seperti \lstlistingname~\ref{lst:301518} akan disimpan dalam \textit{directory} \texttt{Undefined/Pinaceae/Pinus/Pinus\ pinea\ L.}.

\scriptsize
\lstinputlisting[language=xml, numbers=left, numberstyle=\tiny, caption=Contoh citra tanpa informasi jenis, showstringspaces=false, label=lst:301518]{script/301518.xml}
\normalsize

Operasi pencarian semua berkas pada \textit{directory} tertentu juga bermanfaat ketika diperlukan operasi pengolahan citra seperti \texttt{resize} atau \texttt{rescale}. Fitur \textit{Histogram of Oriented Gradients} seperti akan dijelaskan pada sub bab \ref{sec:hog} sangat dipengaruhi oleh ukuran citra. Sehingga citra yang akan dianalisis harus dalam ukuran yang sama untuk mengetahui aspek pembedanya, yaitu obyek yang terdapat di dalam citra. Tentu tidak efisien jika operasi \texttt{resize} atau \texttt{rescale} harus dilakukan secara manual.
