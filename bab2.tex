\chapter{Dasar Pemrograman Python}
\section{Pendahuluan}
Bahasa pemrograman Python memiliki 4 sifat dasar berikut\footnote{\url{https://www.tutorialspoint.com/python/index.htm}}.
\begin{enumerate}
  \item \textit{Interpreter}. Python diproses oleh \textit{interpreter}, sehingga tidak perlu dikompilasi untuk menjalankannya. Hal ini seperti dijumpai pada bahasa pemrograman PHP yang sangat populer itu.
  \item Interaktif. Anda dapat berinteraksi denga Python dengan memberikannya perintah satu per satu melalui Python \texttt{shell}. Setiap perintah yang diberikan langsung akan direspon.
  \item Berorientasi obyek. Ada semacam slogan bahwa '''\textit{Everything is object in Python}'''. Seperti telah dipahami melalu kuliah Rekayasa Perangkat Lunak, orientasi obyek menyebabkan variabel dan fungsi (sering disebut sebagai \textit{state} dan \textit{behavior}) terkemas dalam sebuah obyek, sehingga memudahkan pengelolaan variabel. Fungsi yang melekat pada sebuah obyek juga dapat diturunkan dari satu obyek ke obyek lain sehingga tidak perlu dideklarasi ulang.
  \item Bahasa pemrograman untuk pemula. Hal ini disebabkan karena Python sangat sederhana, tidak memerlukan banyak deklarasi yang seringkali menyulitkan, bahkan menakutkan bagi pemula. Selain itu, Python juga mendukung pengembangan aplikasi untuk banyak \textit{platform}, dari aplikasi \textit{embedded} hingga \textit{web} dan \textit{mobile}. 
\end{enumerate}

Untuk sifat dasar pertama dan kedua, dapat dilihat ilustrasinya di \figurename~\ref{fig:interpreter}. Dalam \figurename~\ref{fig:interpreter}, Python \texttt{shell} dipanggil dengan perintah \texttt{python3} Hal tersebut disebabkan karena Ubuntu (yang sedang digunakan adalah Ubuntu 18.04) secara \textit{default} menyertakan Python versi 2.x. Sedangkan untuk Python versi 3.x harus dijalankan dengan perintah \texttt{python3}. Di \figurename~\ref{fig:interpreter} terlihat bahwa ada dua perintah yang diberikan secara berurutan. Tetapi, Python akan meresponnya satu per satu. Sedangkan untuk keluar dari Python \texttt{shell}, berikan perintah \texttt{exit()}.

\begin{figure}[h!]
  \begin{center}
    \includegraphics[scale=.5]{pics/interpreter.png}
    \caption{Python \texttt{shell} sedang menerima perintah}
    \label{fig:interpreter}
  \end{center}
\end{figure}

Untuk sifat dasar ketiga dapat diilustrasikan melalui \figurename~\ref{fig:obyek}. Kita dapat mengetahui jenis obyek dari variabel \texttt{a} dengan fungsi \texttt{type(a)}. Sedangkan untuk melihat fungsi dan variabel apa saja yang terkandung pada variabel \texttt{a}, kita dapat menggunakan fungsi \texttt{dir(a)}.

\begin{figure}[h!]
  \begin{center}
    \includegraphics[scale=.5]{pics/interpreter2.png}
    \caption{Variabel \texttt{a} sebagai obyek}
    \label{fig:obyek}
  \end{center}
\end{figure}

Dalam \figurename~\ref{fig:obyek} terlihat ada entitas yang diawali dan/atau diakhir dengan karakter dua \textit{underscore} ('\_\_') atau sering disebut sebagi \textit{dunder}\footnote{\url{https://dbader.org/blog/meaning-of-underscores-in-python}} (\textit{double undescore}) oleh komunitas pemrogram Python.
