%-----------------------------------------------------------------------------%
\chapter*{Kata Pengantar}
%-----------------------------------------------------------------------------%
Diktat kuliah ini hanya merupakan pelengkap agar mahasiswa dapat lebih mudah memahami materi pengolahan citra digital. Penggunaan ilustrasi lain dari perangkat lunak berbayar dapat saja diberikan. Tetapi, karena pertimbangan kemandirian dan lisensi, maka saya memutuskan untuk menyusun diktat ini berbasis pada pustaka berlisensi publik dan berbasis bahasa pemrograman Python, \texttt{scikit-image}. Python dipertimbangkan karena banyak pustaka ilmiah yang sudah umum digunakan dan terus dikembangkan yang berbasis pada Python. Dalam pengolahan citra, selain scikit-image, ada juga OpenCV untuk \textit{Computer Vision}. Dalam pembelajaran mesin, \texttt{scikit-learn} adalah pustaka yang juga banyak digunakan. Bahkan \texttt{tensorflow}, pustaka yang banyak digunakan dalam penelitian \textit{deep learning} juga berbasis pada Python. Saya yakin, dengan mempelajari diktat ini, mahasiswa mampu mandiri dalam penguasaan bahasa pemrograman Python yang pada akhirnya mampu membuat mahasiwa lebih adaptif terhadap pustaka berbasis python, baik untuk tujuan ilmiah maupun bisnis. Mahasiswapun diharapkan menjadi lebih kreatif dalam melakukan penelitian hingga mengembangkan produk perangkat lunak, maupun prototipe perangkat keras cerdas berbasis Python tanpa harus terbebani masalah lisensi.

Secara umum, diktat ini dibagi ke dalam bagian pendahuluan yang membahas tentang sejarah singkat Python yang dilanjutkan ke bagian instalasi. Instalasi ini, meskipun sangat sederhana, terutama pada sistem operasi Linux, dapat menjadi sangat merepotkan bagi beberapa mahasiswa, terutama ketika mereka menggunakan sistem operasi Windows. Karena itu, instalasi akan dilakukan di sistem operasi Windows. Bagian selanjutnya adalah dasar-dasar pemrograman Python, terutama struktur data (\texttt{list}, \texttt{tuple} dan \texttt{dictionary}), interaksi dengan \textit{file}, hingga mempelajari penggunaan fungsi yang terdapat dalam pustaka tertentu. Sedangkan bagian terkahir dari diktat ini akan sepenuhnya diisi dengan fitur pustaka \texttt{scikit-image}, yang saat diktat ini disusun berada pada rilis 0.16.

Diktat ini banyak menggunakan sumber dari situs web dan akan disampaikan secara detil alamat sumber tersebut dalam diktat. Diharapkan, mahasiswa tidak takut mencoba karena ada begitu banyak sumber yang dapat digunakan untuk belajar. Hanya kesungguhan kitalah yang akan menjadi pembeda. Akhirnya, selamat mencoba pengalaman baru. 

\vspace*{2cm}
\begin{flushright}
\selectlanguage{bahasai}
Serpong, \today\\[0.1cm]
\vspace*{1cm}
Dr. Arya Adhyaksa Waskita

\end{flushright}
